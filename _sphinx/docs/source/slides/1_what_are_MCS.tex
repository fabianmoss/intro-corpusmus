\section{\thesection.~What are Musical Corpus Studies}

\begin{frame}{A possible definition}
    \begin{quote}
        \textbf{Corpus studies.} Corpus studies are possibly the most common type of project in
        \alert{computational musicology}. A corpus study uses \alert{software} to analyze \alert{statistical patterns}
        in a large collection ― corpus ― of \alert{musical works}. It is, essentially, descriptive statistics
        for musical data. Like text-based corpus studies, musical corpus studies 
        often use \alert{n-gram} and \alert{cluster analysis} methods. 
        Unlike text-based corpus studies, musical corpus
        studies often involve \alert{Markov models} ― probability analyses for progressions in time,
        such as how likely is [a] C-major chord to progress to a D-minor chord in a piece in the
        key of A minor.~\citep{Schaffer2016}
    \end{quote}

    \pause

    Which of these terms are not clear?
\end{frame}

\begin{frame}{History of Musical Corpus Studies}

    Short overview~\citep[after][]{Temperley2013a}:
    \begin{itemize}
        \item Jeppesen (1927): counts of contrapuntal features in Palestrina
        \item Cohen (1962), Youngblood (1958): statistics and information theory 
        \item Meyer (1956, 1967, 1989): information theory and relation to psychology
        \item Krumhansl (1990): algorithm for key finding, relation to music perception
        \item \citet{Huron2001,Huron2006,Huron2016}: corpus studies on melodies and voice leading
    \end{itemize}

    \pause

    Important tools:
    \begin{itemize}
        \item Humdrum (Huron, 1999)
        \item music21 \citep{Cuthbert2010}
    \end{itemize}
\end{frame}

\begin{frame}{Data-related issues}

    \begin{itemize}
        \item \citet{London2013}: representativity
        \item \citet{Pugin2015}: encoding vs OMR
        \item \citet{Neuwirth2018}: balancedness and biases
    \end{itemize}
    
\end{frame}