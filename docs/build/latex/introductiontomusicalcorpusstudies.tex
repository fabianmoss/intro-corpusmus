%% Generated by Sphinx.
\def\sphinxdocclass{report}
\documentclass[letterpaper,10pt,english]{sphinxmanual}
\ifdefined\pdfpxdimen
   \let\sphinxpxdimen\pdfpxdimen\else\newdimen\sphinxpxdimen
\fi \sphinxpxdimen=.75bp\relax

\PassOptionsToPackage{warn}{textcomp}
\usepackage[utf8]{inputenc}
\ifdefined\DeclareUnicodeCharacter
% support both utf8 and utf8x syntaxes
  \ifdefined\DeclareUnicodeCharacterAsOptional
    \def\sphinxDUC#1{\DeclareUnicodeCharacter{"#1}}
  \else
    \let\sphinxDUC\DeclareUnicodeCharacter
  \fi
  \sphinxDUC{00A0}{\nobreakspace}
  \sphinxDUC{2500}{\sphinxunichar{2500}}
  \sphinxDUC{2502}{\sphinxunichar{2502}}
  \sphinxDUC{2514}{\sphinxunichar{2514}}
  \sphinxDUC{251C}{\sphinxunichar{251C}}
  \sphinxDUC{2572}{\textbackslash}
\fi
\usepackage{cmap}
\usepackage[T1]{fontenc}
\usepackage{amsmath,amssymb,amstext}
\usepackage{babel}



\usepackage{times}
\expandafter\ifx\csname T@LGR\endcsname\relax
\else
% LGR was declared as font encoding
  \substitutefont{LGR}{\rmdefault}{cmr}
  \substitutefont{LGR}{\sfdefault}{cmss}
  \substitutefont{LGR}{\ttdefault}{cmtt}
\fi
\expandafter\ifx\csname T@X2\endcsname\relax
  \expandafter\ifx\csname T@T2A\endcsname\relax
  \else
  % T2A was declared as font encoding
    \substitutefont{T2A}{\rmdefault}{cmr}
    \substitutefont{T2A}{\sfdefault}{cmss}
    \substitutefont{T2A}{\ttdefault}{cmtt}
  \fi
\else
% X2 was declared as font encoding
  \substitutefont{X2}{\rmdefault}{cmr}
  \substitutefont{X2}{\sfdefault}{cmss}
  \substitutefont{X2}{\ttdefault}{cmtt}
\fi


\usepackage[Bjarne]{fncychap}
\usepackage[,numfigreset=1,mathnumfig]{sphinx}

\fvset{fontsize=\small}
\usepackage{geometry}


% Include hyperref last.
\usepackage{hyperref}
% Fix anchor placement for figures with captions.
\usepackage{hypcap}% it must be loaded after hyperref.
% Set up styles of URL: it should be placed after hyperref.
\urlstyle{same}

\addto\captionsenglish{\renewcommand{\contentsname}{Content}}

\usepackage{sphinxmessages}
\setcounter{tocdepth}{2}



\title{Introduction to Musical Corpus Studies}
\date{Oct 26, 2020}
\release{0.0.1}
\author{Fabian C.\@{} Moss}
\newcommand{\sphinxlogo}{\vbox{}}
\renewcommand{\releasename}{Release}
\makeindex
\begin{document}

\pagestyle{empty}
\sphinxmaketitle
\pagestyle{plain}
\sphinxtableofcontents
\pagestyle{normal}
\phantomsection\label{\detokenize{index::doc}}


\noindent{\hspace*{\fill}\sphinxincludegraphics[width=1.000\linewidth]{{abstract_bg}.jpg}\hspace*{\fill}}

\begin{sphinxadmonition}{warning}{Warning:}
This material is still (heavily) under construction and might change throughout the course!

You can help improving the course and \sphinxhref{mailto:fabian.moss@epfl.ch}{let me know} about any errors and inconsistencies that you find
or suggest other ways of improving the course.
\end{sphinxadmonition}
\subsubsection*{Welcome!}

These pages present the content of the course “Introduction to Musical Corpus Studies” at the \sphinxhref{http://musikwissenschaft.phil-fak.uni-koeln.de/}{Institute of Musicology},
given at \sphinxhref{https://uni-koeln.de/}{University of Cologne} in Fall 2020.

In the last two decades \sphinxstyleemphasis{Musical Corpus Studies} evolved from a niche discipline into a veritable research area.
The growing availability of digital and digitized musical data as well as the application and development of modern
methodologies from computer science, machine learning, and data science cast new light on old musicological questions
and generate entirely novel approaches to empirical music research.

Moreover, the general methodological and epistemological approach of Musical Corpus Studies allows to transcend traditional
intra\sphinxhyphen{}musicological boundaries between its sub\sphinxhyphen{}disciplintes (historical/systematic/ethnological/…) without sacrificing the
respective specific viewpoints and perspectives.

This course offers a fundamental and practical introduction into these topics.
It demonstrates, explores, and critically reflects central thematic areas and methods by means of a number of case studies.
In the engagement with these topics the course also introduces elementary methods from natural language and music processing,
as well as statistics, data analysis and visualization.

The course is aimed at students at the undergraduate level who have little or no empirical background and are curious
about quantitative approaches to musicology.


\chapter{Organization}
\label{\detokenize{1_orga:organization}}\label{\detokenize{1_orga::doc}}

\section{About this course}
\label{\detokenize{1_orga:about-this-course}}
Programing introductions often boring.
A lot of time lost in introducing basic concepts and techniques (important!)
but quite remote from actual (!) applications. Examples are usually “toy examples”
that work well, but the transition to real\sphinxhyphen{}world applications is difficult.
Of course, the example studies discussed in this course work well, too.
However, they are without exception taken from peer\sphinxhyphen{}reviewed, published, open access articles.
They thus reflect actual, recent research questions that reflect current research.

This course takes thus the opposite approach to “toy examples”. We will not introduce many specific
programing concepts. The course rather showcases what is possible with musical corpus studies.
If this sparks your interest, it will be much easier to pick up the basics for yourself,
knowing what they are \sphinxstyleemphasis{for} and being motivated intrinsically.
If you are not particularly interested in doing this kind of work yourself,
you will still see a broad range of applications that are much more useful to you than
learning (or not learning) Python basics.


\section{Overview}
\label{\detokenize{1_orga:overview}}

\begin{savenotes}\sphinxattablestart
\centering
\begin{tabulary}{\linewidth}[t]{|T|T|T|T|T|}
\hline
\sphinxstyletheadfamily 
No.
&\sphinxstyletheadfamily 
Date
&\sphinxstyletheadfamily 
Time
&\sphinxstyletheadfamily 
Room
&\sphinxstyletheadfamily 
Topics
\\
\hline
1
&
Fr., 13.11.2020
&
16:00\sphinxhyphen{}17:20 Uhr
&
Neuer Seminarraum 1.315
&
Introduction / Background
\\
\hline
2
&&
17:40\sphinxhyphen{}19:00 Uhr
&&
Folk Songs, Melodies, Pitches and Intervals  frequencies, mean, variance
\\
\hline
3
&
Sa., 14.11.2020
&
09:00\sphinxhyphen{}10:20 Uhr
&
Neuer Seminarraum 1.315
&
Jazz Solos, Melodies, Regular Expressions
\\
\hline
4
&&
10:40\sphinxhyphen{}12:00 Uhr
&&
Beethoven’s string quartets, harmony, \(n\)\sphinxhyphen{}grams, Markov models
\\
\hline&&
12:00\sphinxhyphen{}13:00 Uhr
&&
Lunch Break
\\
\hline
5
&&
13:00\sphinxhyphen{}14:20 Uhr
&&
Pop Charts Billboard 100, harmony, Clustering, \(k\)\sphinxhyphen{}means, {[}Hidden Markov Models{]}
\\
\hline
6
&&
14:40\sphinxhyphen{}16:00 Uhr
&&
Group work
\\
\hline
7
&
Fr., 11.12.2020
&
10:00\sphinxhyphen{}11:20 Uhr
&
Alter Seminarraum 1.408
&
Cadences in Renaissance Polyphony  with guest researcher \sphinxhref{https://www.haverford.edu/users/rfreedma}{Richard Freedman}
\\
\hline
8
&&
11:40\sphinxhyphen{}13:00 Uhr
&&
Brazilian Choro, harmony, form, context\sphinxhyphen{}Free Grammars
\\
\hline
9
&
Sa., 12.12.2020
&
09:00\sphinxhyphen{}10:20 Uhr
&
Neuer Seminarraum 1.315
&
Malian Percussion Music, rhythm, meter
\\
\hline
10
&&
10:40\sphinxhyphen{}12:00 Uhr
&&
Electronic Music 1950\sphinxhyphen{}1990
\\
\hline&&
12:00\sphinxhyphen{}13:00 Uhr
&&
Lunch Break
\\
\hline
11
&&
13:00\sphinxhyphen{}14:20 Uhr
&&
Group work
\\
\hline
12
&&
14:40\sphinxhyphen{}16:00 Uhr
&&
Recapitulation and conclusion
\\
\hline
\end{tabulary}
\par
\sphinxattableend\end{savenotes}


\section{Credits}
\label{\detokenize{1_orga:credits}}
Active participation in this course is compensated with 3 credit points (CPs),
\sphinxhref{https://verwaltung.uni-koeln.de/abteilung21/content/studienangebot/studiengaenge\_u\_\_abschluesse/bachelor\_\_und\_masterstudiengaenge/index\_ger.html}{equivalent to a work load of 90 hours}.
These are distributed as follows: 24 SWS (90 minutes) are allocated to presence in the block seminar.
Additionally, 36 SWS are dedicated to the preparation and follow\sphinxhyphen{}up of the material.
The remainder of 30 SWS goes to the reading of the relevant literature.


\section{Deliverables and Learning objectives}
\label{\detokenize{1_orga:deliverables-and-learning-objectives}}
Course work consists of three parts: preparing the relevant literature (reading),
completing the relevant exercises (group work), and critically engaging with the course materials
in the form of a report written together with your group.
\begin{itemize}
\item {} 
work load management

\item {} 
organization

\end{itemize}
\subsubsection*{Reading}

For each session, the relevant literature is cited in the text.
Careful preparation is required in order to be able to follow the content of the course.
Because the course will mainly talk about methods and general points of musical corpus research,
the content (and musical topic) will mainly be introduced by the literature.

I am aware that the reading workload is relatively high since the course will be taught as a block seminar
and doesn’t spread out over the entire semester. I hope that the fact that the course is finished before the
end of the year compensates for this.
\begin{itemize}
\item {} 
critical reading of scientific literature

\end{itemize}
\subsubsection*{Group work}

At the beginning of the course, you will be randomly assigned to a group.
Together with your group, you will work on a number of exercises during the course,
e.g. in Zoom breakout rooms.
\begin{itemize}
\item {} 
content of the course units

\item {} 
specific musicological and/or methodological questions

\end{itemize}
\subsubsection*{Review}

After the course has ended, your group will be randomly assigned a course topic.
It is your task to write a review/report on this topic.
What did you learn? Which concepts are not clear? Which methods did you (not) understand?
What is missing? How can the textual descriptions be improved? Who in your group did what?
Write about the organization of your group, challenges and benefits.
\begin{itemize}
\item {} 
create issues on GitHub

\item {} 
writing academic reviews

\end{itemize}

Recommended structure:
\begin{enumerate}
\sphinxsetlistlabels{\arabic}{enumi}{enumii}{}{.}%
\item {} 
Introduction: general description and summary of the course and your session in particular.

\item {} 
…

\end{enumerate}

\begin{sphinxadmonition}{important}{Important:}
Submit your report by \sphinxstylestrong{31 January 2021} to \sphinxhref{mailto:fabian.moss@epfl.ch}{fabian.moss@epfl.ch}.
\end{sphinxadmonition}


\chapter{Introduction}
\label{\detokenize{2_introduction:introduction}}\label{\detokenize{2_introduction::doc}}
\noindent{\hspace*{\fill}\sphinxincludegraphics[width=1.000\linewidth]{{pattern}.jpg}\hspace*{\fill}}


\section{What are Musical Corpus Studies?}
\label{\detokenize{2_introduction:what-are-musical-corpus-studies}}
tbc… (text from diss?)

\begin{figure}[htbp]
\centering
\capstart

\noindent\sphinxincludegraphics[width=0.800\linewidth]{{style_analysis}.jpg}
\caption{Hierarchy of musical style analyis after \sphinxcite{bibliography:meyer1989}, from \sphinxcite{bibliography:jan2007}.}\label{\detokenize{2_introduction:id11}}\end{figure}


\section{Epistemological goals}
\label{\detokenize{2_introduction:epistemological-goals}}
tbc…


\section{Issues}
\label{\detokenize{2_introduction:issues}}
tbc \sphinxcite{bibliography:cook2006}\sphinxcite{bibliography:honing2006}\sphinxcite{bibliography:huron2013}\sphinxcite{bibliography:neuwirth2016}\sphinxcite{bibliography:pugin2005}\sphinxcite{bibliography:schaffer2016}\sphinxcite{bibliography:temperley2013}\sphinxcite{bibliography:volk2011}


\section{MCS and traditional musicology}
\label{\detokenize{2_introduction:mcs-and-traditional-musicology}}
tbc


\section{Basic representations}
\label{\detokenize{2_introduction:basic-representations}}\begin{itemize}
\item {} 
tones, notes

\item {} 
(tonal/neutral) pitch classes

\item {} 
meter (hierarchy)

\end{itemize}



\begin{sphinxthebibliography}{VWvK11}
\bibitem[Coo06]{bibliography:cook2006}
Nicholas Cook. Border Crossings: A Commentary on Henkjan Honing’s “On the Growing Role of Observation, Formalization and Experimental Method in Musicology”. \sphinxstyleemphasis{Empirical Musicology Review}, 1(1):7\textendash{}11, 2006.
\bibitem[Hon06]{bibliography:honing2006}
Henkjan Honing. On the Growing Role of Observation, Formalization and Experimental Method in Musicology. \sphinxstyleemphasis{Empirical Musicology Review}, 1(1):2\textendash{}6, 2006.
\bibitem[Hur13]{bibliography:huron2013}
David Huron. On the Virtuous and the Vexatious in an Age of Big Data. \sphinxstyleemphasis{Music Perception: An Interdisciplinary Journal}, 31(1):4\textendash{}9, 2013. \sphinxhref{https://doi.org/10.1525/mp.2013.31.1.4}{doi:10.1525/mp.2013.31.1.4}.
\bibitem[Jan07]{bibliography:jan2007}
Steven Jan. \sphinxstyleemphasis{The Memetics of Music: A Neo\sphinxhyphen{}Darwinian View of Musical Structure and Culture}. Ashgate, 2007.
\bibitem[Mey89]{bibliography:meyer1989}
Leonard B. Meyer. \sphinxstyleemphasis{Style and Music. Theory, History, and Ideology}. University of Chicago Press, 1989.
\bibitem[NR16]{bibliography:neuwirth2016}
Markus Neuwirth and Martin Rohrmeier. Wie wissenschaftlich muss Musiktheorie sein? Chancen und Herausforderungen musikalischer Korpusforschung. \sphinxstyleemphasis{Zeitschrift der Gesellschaft für Musiktheorie {[}Journal of the German\sphinxhyphen{}speaking Society of Music Theory{]}}, 13(2):171\textendash{}193, 2016. \sphinxhref{https://doi.org/10.31751/915}{doi:10.31751/915}.
\bibitem[Pug15]{bibliography:pugin2005}
Laurent Pugin. The challenge of data in digital musicology. \sphinxstyleemphasis{Frontiers in Digital Humanities}, 2(4):1\textendash{}3, 2015. \sphinxhref{https://doi.org/10.3389/fdigh.2015.00004}{doi:10.3389/fdigh.2015.00004}.
\bibitem[Sch16]{bibliography:schaffer2016}
Kris Schaffer. What is computational musicology? online, 2016. URL: \sphinxurl{https://medium.com/@krisshaffer/what-is-computational-musicology-f25ee0a65102}.
\bibitem[TV13]{bibliography:temperley2013}
David Temperley and Leigh VanHandel. Introduction to the Special Issue on Corpus Methods. \sphinxstyleemphasis{Music Perception: An Interdisciplinary Journal}, 31(1):1\textendash{}3, 2013.
\bibitem[VWvK11]{bibliography:volk2011}
Anja Volk, Frans Wiering, and Peter van Kranenburg. Unfolding the Potential of Computational Musicology. In R. J. Jorna, K. Liu, and N. R. Faber, editors, \sphinxstyleemphasis{Proceedings of the Thirteenth International Conference on Informatics and Semiotics in Organisations: Problems and Possibilities of Computational Humanities}, 978\textendash{}94. Fryske Academy, 2011.
\end{sphinxthebibliography}



\renewcommand{\indexname}{Index}
\printindex
\end{document}