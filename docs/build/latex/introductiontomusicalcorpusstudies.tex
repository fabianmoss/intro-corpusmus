%% Generated by Sphinx.
\def\sphinxdocclass{report}
\documentclass[letterpaper,10pt,english]{sphinxmanual}
\ifdefined\pdfpxdimen
   \let\sphinxpxdimen\pdfpxdimen\else\newdimen\sphinxpxdimen
\fi \sphinxpxdimen=.75bp\relax

\PassOptionsToPackage{warn}{textcomp}
\usepackage[utf8]{inputenc}
\ifdefined\DeclareUnicodeCharacter
% support both utf8 and utf8x syntaxes
  \ifdefined\DeclareUnicodeCharacterAsOptional
    \def\sphinxDUC#1{\DeclareUnicodeCharacter{"#1}}
  \else
    \let\sphinxDUC\DeclareUnicodeCharacter
  \fi
  \sphinxDUC{00A0}{\nobreakspace}
  \sphinxDUC{2500}{\sphinxunichar{2500}}
  \sphinxDUC{2502}{\sphinxunichar{2502}}
  \sphinxDUC{2514}{\sphinxunichar{2514}}
  \sphinxDUC{251C}{\sphinxunichar{251C}}
  \sphinxDUC{2572}{\textbackslash}
\fi
\usepackage{cmap}
\usepackage[T1]{fontenc}
\usepackage{amsmath,amssymb,amstext}
\usepackage{babel}



\usepackage{times}
\expandafter\ifx\csname T@LGR\endcsname\relax
\else
% LGR was declared as font encoding
  \substitutefont{LGR}{\rmdefault}{cmr}
  \substitutefont{LGR}{\sfdefault}{cmss}
  \substitutefont{LGR}{\ttdefault}{cmtt}
\fi
\expandafter\ifx\csname T@X2\endcsname\relax
  \expandafter\ifx\csname T@T2A\endcsname\relax
  \else
  % T2A was declared as font encoding
    \substitutefont{T2A}{\rmdefault}{cmr}
    \substitutefont{T2A}{\sfdefault}{cmss}
    \substitutefont{T2A}{\ttdefault}{cmtt}
  \fi
\else
% X2 was declared as font encoding
  \substitutefont{X2}{\rmdefault}{cmr}
  \substitutefont{X2}{\sfdefault}{cmss}
  \substitutefont{X2}{\ttdefault}{cmtt}
\fi


\usepackage[Bjarne]{fncychap}
\usepackage{sphinx}

\fvset{fontsize=\small}
\usepackage{geometry}


% Include hyperref last.
\usepackage{hyperref}
% Fix anchor placement for figures with captions.
\usepackage{hypcap}% it must be loaded after hyperref.
% Set up styles of URL: it should be placed after hyperref.
\urlstyle{same}
\addto\captionsenglish{\renewcommand{\contentsname}{Content}}

\usepackage{sphinxmessages}
\setcounter{tocdepth}{0}



\title{Introduction to Musical Corpus Studies}
\date{Jul 29, 2020}
\release{0.0.1}
\author{Fabian C.\@{} Moss}
\newcommand{\sphinxlogo}{\vbox{}}
\renewcommand{\releasename}{Release}
\makeindex
\begin{document}

\pagestyle{empty}
\sphinxmaketitle
\pagestyle{plain}
\sphinxtableofcontents
\pagestyle{normal}
\phantomsection\label{\detokenize{index::doc}}
\begin{figure}[htbp]
\centering
\capstart

\noindent\sphinxincludegraphics[width=1.000\linewidth]{{abstract_bg}.jpg}
\caption{Image by \sphinxhref{https://unsplash.com/@pixeldebris?utm\_source=unsplash\&amp;utm\_medium=referral\&amp;utm\_content=creditCopyText}{Victoria Alexander}
on \sphinxhref{https://unsplash.com/s/photos/pattern?utm\_source=unsplash\&amp;utm\_medium=referral\&amp;utm\_content=creditCopyText}{Unsplash}.}\label{\detokenize{index:id1}}\end{figure}



\begin{sphinxadmonition}{warning}{Warning:}
These pages present the content of the course “Introduction to Musical Corpus Studies”,
given at University of Cologne in Fall 2020.

Note that this material is still (heavily) under construction and might change throughout the course!

You can help improving the course and let me know about any errors and inconsistencies that you find
or suggest other ways of improving the course.
\end{sphinxadmonition}
\subsubsection*{Welcome!}

In the last two decades \sphinxstyleemphasis{Musical Corpus Studies} evolved from a niche discipline into a veritable research area.
The growing availability of digital and digitized musical data as well as the application and development of modern
methodologies from computer science, machine learning, and data science cast new light on old musicological questions
and generate entirely novel approaches to empirical music research.

Moreover, the general methodological and epistemological approach of Musical Corpus Studies allows to transcend traditional
intra\sphinxhyphen{}musicological boundaries between its sub\sphinxhyphen{}disciplintes (historical/systematic/ethnological/…) without sacrificing the
respective specific viewpoints and perspectives.

This course offers a fundamental and practical introduction into these topics.
It demonstrates, explores, and critically reflects central thematic areas and methods by means of a number of case studies.
Among the contents are:
\begin{itemize}
\item {} 
Beethoven’s string quartets

\item {} 
19th century piano music

\item {} 
Popular music charts

\item {} 
Electronic music 1950\sphinxhyphen{}1990

\item {} 
Brazilian Choro

\item {} 
Malian percussion music

\item {} 
Jazz solos

\end{itemize}

In the engagement with these topics the course also introduces elementary methods from natural language and music processing,
as well as statistics, data analysis and visualization.


\chapter{Organization}
\label{\detokenize{orga:organization}}\label{\detokenize{orga::doc}}

\section{Schedule}
\label{\detokenize{orga:schedule}}
The following table outlines the schedule and summarizes the contents of this course.


\begin{savenotes}\sphinxattablestart
\centering
\begin{tabulary}{\linewidth}[t]{|T|T|T|T|T|T|T|}
\hline
\sphinxstyletheadfamily 
No.
&\sphinxstyletheadfamily 
Date
&\sphinxstyletheadfamily 
Time
&\sphinxstyletheadfamily 
Room
&\sphinxstyletheadfamily 
Topic
&\sphinxstyletheadfamily 
Corpus
&\sphinxstyletheadfamily 
Methods
\\
\hline
1
&
Fr., 13.11.2020
&
16:00\sphinxhyphen{}17:20 Uhr
&
Neuer Seminarraum 1.315
&
Introduction / Background
&&\\
\hline
2
&&
17:40\sphinxhyphen{}19:00 Uhr
&&
Folk Songs, Melodies, Pitches and Intervals
&
Essen Folk Song Collection
&
frequencies, 
mean, 
variance
\\
\hline
3
&
Sa., 14.11.2020
&
09:00\sphinxhyphen{}10:20 Uhr
&
Neuer Seminarraum 1.315
&
Jazz Solos, Melodies
&
Weimar Jazz Database
&
Regular Expressions
\\
\hline
4
&&
10:40\sphinxhyphen{}12:00 Uhr
&&
Beethoven’s string quartets 
Harmony
&
Annotated Beethoven Corpus
&
\(n\)\sphinxhyphen{}grams, 
Markov models
\\
\hline&&
12:00\sphinxhyphen{}13:00 Uhr
&&
Lunch Break
&&\\
\hline
5
&&
13:00\sphinxhyphen{}14:20 Uhr
&&
Pop Charts Billboard 100, harmony,
&
McGill Billboard Dataset
&
Hidden Markov Models
\\
\hline
6
&&
14:40\sphinxhyphen{}16:00 Uhr
&&
Free group work
&&\\
\hline
7
&
Fr., 11.12.2020
&
10:00\sphinxhyphen{}11:20 Uhr
&
Alter Seminarraum 1.408
&
Brazilian Choro, harmony, form,
&
Choro Songbook Corpus
&
Context\sphinxhyphen{}Free Grammars
\\
\hline
8
&&
11:40\sphinxhyphen{}13:00 Uhr
&&
19th century piano music, harmony
&
DCML Piano Corpus
&
Probabilistic CFGs
\\
\hline
9
&
Sa., 12.12.2020
&
09:00\sphinxhyphen{}10:20 Uhr
&
Neuer Seminarraum 1.315
&
Malian Percussion Music, rhythm, meter
&
Interpersonal Entrainment in Music Performance: 
Malian Jembe
&\\
\hline
10
&&
10:40\sphinxhyphen{}12:00 Uhr
&&
Electronic Music 1950\sphinxhyphen{}1990
&
Curated Corpus of Historical Electronic Music
&\\
\hline&&
12:00\sphinxhyphen{}13:00 Uhr
&&
Lunch Break
&&\\
\hline
11
&&
13:00\sphinxhyphen{}14:20 Uhr
&&
Free group work
&&\\
\hline
12
&&
14:40\sphinxhyphen{}16:00 Uhr
&&
Recapitulation and conclusion
&&\\
\hline
\end{tabulary}
\par
\sphinxattableend\end{savenotes}


\section{Credits}
\label{\detokenize{orga:credits}}
Ich gehe in der Seminarplanung von 12 Semesterwochen à 2 SWS aus, für das gesamte Blockseminar also 24 SWS.
Das Seminar wird mit 3 CP bewertet, was 90 Stunden aktiver Arbeit entspricht.
Davon entfallen 24 SWS an die Präsenzzeit im Seminar plus 48 SWS an Vor\sphinxhyphen{} und Nachbereitung der Seminarsitzungen.
Die verbleibenden 18 SWS sind für die Lektüre der Fachliteratur vorgesehen.


\chapter{Introduction / Background}
\label{\detokenize{background:introduction-background}}\label{\detokenize{background::doc}}

\section{What are Musical Corpus Studies?}
\label{\detokenize{background:what-are-musical-corpus-studies}}
tbc… (text from diss?)


\section{Epistemological goals}
\label{\detokenize{background:epistemological-goals}}
tbc…


\section{Issues}
\label{\detokenize{background:issues}}
tbc


\section{MCS and traditional musicology}
\label{\detokenize{background:mcs-and-traditional-musicology}}
tnc
\subsubsection*{References}
\begin{enumerate}
\sphinxsetlistlabels{\arabic}{enumi}{enumii}{}{.}%
\item {} 
Cook (2006). Border Crossings: A Commentary on Henkjan Honing’s “On the Growing Role of Observation, Formalization and Experimental Method in Musicology”. Empirical Musicology Review 1(1), 7\sphinxhyphen{}11.

\item {} 
Honing (2006). On the Growing Role of Observation, Formalization and Experimental Method in Musicology. Empirical Musicology Review 1(1), 2\sphinxhyphen{}6.

\item {} 
Huron (2013). On the Virtuous and the Vexatious in an Age of Big Data. Music Perception: An Interdisciplinary Journal. 31(1), 4\sphinxhyphen{}9.

\item {} 
Marsden (2016). Music Analysis by Computer: Ontology and Epistemology. In: David Meredith (ed.) Computational Music Analysis. Springer.

\item {} 
Neuwirth \& Rohrmeier (2016). Wie wissenschaftlich muss Musiktheorie sein? Chancen und Herausforderungen musikalischer Korpusforschung. Zeitschrift der Gesellschaft für Musiktheorie 13(2), 171\sphinxhyphen{}193.

\item {} 
Pugin (2015). The Challenge of Data in Digital Musicology. Frontiers in Digital Humanities 2(4), 1\sphinxhyphen{}3.

\item {} 
Schaffer (2016). What is Computational Musicology? \sphinxurl{https://medium.com/@krisshaffer/what-is-computational-musicology-f25ee0a65102}.

\item {} 
Temperley \& VanHandel (2013). Introduction to the Special Issue on Corpus Methods. Music Perception: An Interdisciplinary Journal. 31(1), 1\sphinxhyphen{}3.

\end{enumerate}


\chapter{Folk Songs and the Melodic Arc}
\label{\detokenize{folk_songs:folk-songs-and-the-melodic-arc}}\label{\detokenize{folk_songs::doc}}
Huron… / MusThe Tutorial


\chapter{Solos in the \sphinxstyleemphasis{Weimar Jazz Database}}
\label{\detokenize{jazz_solos:solos-in-the-weimar-jazz-database}}\label{\detokenize{jazz_solos::doc}}
The first project we will have a look at is the \sphinxhref{https://jazzomat.hfm-weimar.de/}{Jazzomat} project.
Transcriptions of Jazz solos.


\section{The Project}
\label{\detokenize{jazz_solos:the-project}}\subsubsection*{References}
\begin{enumerate}
\sphinxsetlistlabels{\arabic}{enumi}{enumii}{}{.}%
\item {} 
Pfleiderer et al. (2017). Inside the Jazzomat: New Perspectives for Jazz Research.

\end{enumerate}


\chapter{Harmony in Beethoven’s String Quartets}
\label{\detokenize{beethoven_harmony:harmony-in-beethoven-s-string-quartets}}\label{\detokenize{beethoven_harmony::doc}}

\section{Access the data}
\label{\detokenize{beethoven_harmony:access-the-data}}
The data lies on the GitHub repository \sphinxhref{https://github.com/DCMLab/ABC}{DCMLab/ABC}.
Either download the \sphinxcode{\sphinxupquote{.tsv}} file directly and open it in pandas or load it from the URL as follows:

\begin{sphinxVerbatim}[commandchars=\\\{\}]
\PYG{k+kn}{import} \PYG{n+nn}{pandas} \PYG{k+kn}{as} \PYG{n+nn}{pd}

\PYG{n}{df} \PYG{o}{=} \PYG{n}{pd}\PYG{o}{.}\PYG{n}{read\PYGZus{}csv}\PYG{p}{(}\PYG{l+s+s2}{\PYGZdq{}}\PYG{l+s+s2}{https://github.com/DCMLab/ABC/corpus.tsv}\PYG{l+s+s2}{\PYGZdq{}}\PYG{p}{,} \PYG{n}{sep}\PYG{o}{=}\PYG{l+s+s2}{\PYGZdq{}}\PYG{l+s+se}{\PYGZbs{}t}\PYG{l+s+s2}{\PYGZdq{}}\PYG{p}{)}
\end{sphinxVerbatim}

The corpus is now stored in the variable \sphinxcode{\sphinxupquote{df}}.


\section{Harmonic Annotations}
\label{\detokenize{beethoven_harmony:harmonic-annotations}}\begin{itemize}
\item {} 
regular expressions

\end{itemize}


\section{Chord Transitions}
\label{\detokenize{beethoven_harmony:chord-transitions}}\begin{itemize}
\item {} 
n\sphinxhyphen{}grams

\end{itemize}


\chapter{Billboard Pop Charts}
\label{\detokenize{billboard:billboard-pop-charts}}\label{\detokenize{billboard::doc}}\begin{itemize}
\item {} 
Schaffer et al. (2020). A cluster analysis of harmony in the McGill Billboard dataset. \sphinxstyleemphasis{Empirical Musicology Review} 14(3\textendash{}4), 146\textendash{}162. \sphinxurl{http://dx.doi.org/10.18061/emr.v14i3-4.5576}

\end{itemize}


\chapter{Brazilian Choro}
\label{\detokenize{choro:brazilian-choro}}\label{\detokenize{choro::doc}}

\chapter{19th century piano music}
\label{\detokenize{piano:th-century-piano-music}}\label{\detokenize{piano::doc}}

\chapter{Malian Percussion Music}
\label{\detokenize{mali_percussion:malian-percussion-music}}\label{\detokenize{mali_percussion::doc}}

\chapter{Electronic Music 1950\sphinxhyphen{}1990}
\label{\detokenize{electronic:electronic-music-1950-1990}}\label{\detokenize{electronic::doc}}

\chapter{Conclusion}
\label{\detokenize{conclusion:conclusion}}\label{\detokenize{conclusion::doc}}
Final thoughts, critical discussion…

{[}Some image{]}



\renewcommand{\indexname}{Index}
\printindex
\end{document}