%% Generated by Sphinx.
\def\sphinxdocclass{report}
\documentclass[letterpaper,10pt,english]{sphinxmanual}
\ifdefined\pdfpxdimen
   \let\sphinxpxdimen\pdfpxdimen\else\newdimen\sphinxpxdimen
\fi \sphinxpxdimen=.75bp\relax

\PassOptionsToPackage{warn}{textcomp}
\usepackage[utf8]{inputenc}
\ifdefined\DeclareUnicodeCharacter
% support both utf8 and utf8x syntaxes
  \ifdefined\DeclareUnicodeCharacterAsOptional
    \def\sphinxDUC#1{\DeclareUnicodeCharacter{"#1}}
  \else
    \let\sphinxDUC\DeclareUnicodeCharacter
  \fi
  \sphinxDUC{00A0}{\nobreakspace}
  \sphinxDUC{2500}{\sphinxunichar{2500}}
  \sphinxDUC{2502}{\sphinxunichar{2502}}
  \sphinxDUC{2514}{\sphinxunichar{2514}}
  \sphinxDUC{251C}{\sphinxunichar{251C}}
  \sphinxDUC{2572}{\textbackslash}
\fi
\usepackage{cmap}
\usepackage[T1]{fontenc}
\usepackage{amsmath,amssymb,amstext}
\usepackage{babel}



\usepackage{times}
\expandafter\ifx\csname T@LGR\endcsname\relax
\else
% LGR was declared as font encoding
  \substitutefont{LGR}{\rmdefault}{cmr}
  \substitutefont{LGR}{\sfdefault}{cmss}
  \substitutefont{LGR}{\ttdefault}{cmtt}
\fi
\expandafter\ifx\csname T@X2\endcsname\relax
  \expandafter\ifx\csname T@T2A\endcsname\relax
  \else
  % T2A was declared as font encoding
    \substitutefont{T2A}{\rmdefault}{cmr}
    \substitutefont{T2A}{\sfdefault}{cmss}
    \substitutefont{T2A}{\ttdefault}{cmtt}
  \fi
\else
% X2 was declared as font encoding
  \substitutefont{X2}{\rmdefault}{cmr}
  \substitutefont{X2}{\sfdefault}{cmss}
  \substitutefont{X2}{\ttdefault}{cmtt}
\fi


\usepackage[Bjarne]{fncychap}
\usepackage[,numfigreset=1,mathnumfig]{sphinx}

\fvset{fontsize=\small}
\usepackage{geometry}


% Include hyperref last.
\usepackage{hyperref}
% Fix anchor placement for figures with captions.
\usepackage{hypcap}% it must be loaded after hyperref.
% Set up styles of URL: it should be placed after hyperref.
\urlstyle{same}

\addto\captionsenglish{\renewcommand{\contentsname}{Content}}

\usepackage{sphinxmessages}
\setcounter{tocdepth}{2}



\title{Introduction to Musical Corpus Studies}
\date{Nov 05, 2020}
\release{0.0.1}
\author{Fabian C.\@{} Moss}
\newcommand{\sphinxlogo}{\vbox{}}
\renewcommand{\releasename}{Release}
\makeindex
\begin{document}

\pagestyle{empty}
\sphinxmaketitle
\pagestyle{plain}
\sphinxtableofcontents
\pagestyle{normal}
\phantomsection\label{\detokenize{index::doc}}


\noindent{\hspace*{\fill}\sphinxincludegraphics[width=1.000\linewidth]{{abstract_bg}.jpg}\hspace*{\fill}}

\begin{sphinxadmonition}{warning}{Warning:}
This material is still (heavily) under construction and might change throughout the course!

You can help improving the course and \sphinxhref{mailto:fabian.moss@epfl.ch}{let me know} about any errors and inconsistencies that you find
or suggest other ways of improving the course.
\end{sphinxadmonition}
\subsubsection*{Welcome!}

These pages present the content of the course “Introduction to Musical Corpus Studies” at the \sphinxhref{http://musikwissenschaft.phil-fak.uni-koeln.de/}{Institute of Musicology},
given at \sphinxhref{https://uni-koeln.de/}{University of Cologne} in Fall 2020.

In the last two decades \sphinxstyleemphasis{Musical Corpus Studies} evolved from a niche discipline into a veritable research area.
The growing availability of digital and digitized musical data as well as the application and development of modern
methodologies from computer science, machine learning, and data science cast new light on old musicological questions
and generate entirely novel approaches to empirical music research.

Moreover, the general methodological and epistemological approach of Musical Corpus Studies allows to transcend traditional
intra\sphinxhyphen{}musicological boundaries between its sub\sphinxhyphen{}disciplintes (historical/systematic/ethnological/…) without sacrificing the
respective specific viewpoints and perspectives.

This course offers a fundamental and practical introduction into these topics.
It demonstrates, explores, and critically reflects central thematic areas and methods by means of a number of case studies.
In the engagement with these topics the course also introduces elementary methods from natural language and music processing,
as well as statistics, data analysis and visualization.

The course is aimed at students at the undergraduate level who have little or no empirical background and are curious
about quantitative approaches to musicology.


\chapter{Organization}
\label{\detokenize{1_orga:organization}}\label{\detokenize{1_orga::doc}}

\section{About this course}
\label{\detokenize{1_orga:about-this-course}}
This course aims at providing an example\sphinxhyphen{}based introduction to the rapidly developing field of Musical Corpus Studies (MCS).
Introducing a field that relies equally on musicological domain knowledge and skills in computational and statistical methods
faces obvious challenges: while most people interested in this field come with a background in either area,
few people are versed in both, and it can take years to bridge the musicological\sphinxhyphen{}computational gap.

In particular, systematic introductions to programming or specific musicological topics can be at times quite arduous, even boring,
because it takes a long time to proceed from learning basic concepts to acually interesting problems.
The problems and “toy examples” that are presented to introduce the basic concepts are necessarily remote from
real\sphinxhyphen{}world applications and challenging research problems.

This course takes an alternative route.
It does not start with an introduction to the programming language \sphinxhref{http://python.org/}{Python}
(which will be used throughout to carry out the computational analyses)
but rather showcases a number of recent corpus studies that take on musicological research questions.
The focus thus lies in understanding how aspects of music can be studied with computational methods
and by analyzing musical corpora.

If this sparks your interest, it will be much easier to pick up the basics for yourself,
knowing what they are \sphinxstyleemphasis{for} and being motivated intrinsically.
If you are not particularly interested in doing this kind of work yourself,
you will still see a broad range of applications that are much more useful to you than
learning (or not learning) programming basics.


\section{Overview}
\label{\detokenize{1_orga:overview}}
This year’s course takes place on two weekends (13\sphinxhyphen{}14 November and 11\sphinxhyphen{}12 December 2020),
comprising twelve sessions in total. The topics cover a broad range of musicological topics,
from folk melodies and Jazz solos, over harmonies in Beethoven’s string
quartets and 20th century Pop music, to Renaissance candences
and metric patterns in Malian drum music (see \hyperref[\detokenize{1_orga:tab-overview}]{Table \ref{\detokenize{1_orga:tab-overview}}}).


\begin{savenotes}\sphinxattablestart
\centering
\phantomsection\label{\detokenize{1_orga:tab-overview}}\nobreak
\begin{tabulary}{\linewidth}[t]{|T|T|T|T|}
\hline
\sphinxstyletheadfamily 
No.
&\sphinxstyletheadfamily 
Date
&\sphinxstyletheadfamily 
Time
&\sphinxstyletheadfamily 
Topics
\\
\hline
1
&
Fr., 13.11.2020
&
16:00\sphinxhyphen{}17:20 Uhr
&
Introduction / Background
\\
\hline
2
&&
17:40\sphinxhyphen{}19:00 Uhr
&
Melody I: Pitches, intervals, folk song melodies
\\
\hline
3
&
Sa., 14.11.2020
&
09:00\sphinxhyphen{}10:20 Uhr
&
Melody II: Jazz solos
\\
\hline
4
&&
10:40\sphinxhyphen{}12:00 Uhr
&
Harmony I: Beethoven’s string quartets
\\
\hline&&
12:00\sphinxhyphen{}13:00 Uhr
&
\sphinxstyleemphasis{Lunch Break}
\\
\hline
5
&&
13:00\sphinxhyphen{}14:20 Uhr
&
Group work
\\
\hline
6
&&
14:40\sphinxhyphen{}16:00 Uhr
&
Harmony II: Pop charts (Billboard 100)
\\
\hline
7
&
Fr., 11.12.2020
&
10:00\sphinxhyphen{}11:20 Uhr
&
Harmony III: Cadences in Renaissance polyphony (guest: \sphinxhref{https://www.haverford.edu/users/rfreedma}{Richard Freedman})
\\
\hline
8
&&
11:40\sphinxhyphen{}13:00 Uhr
&
Harmony IV \& Form: Brazilian Choro
\\
\hline
9
&
Sa., 12.12.2020
&
09:00\sphinxhyphen{}10:20 Uhr
&
Rhythm \& Meter: Malian percussion music
\\
\hline
10
&&
10:40\sphinxhyphen{}12:00 Uhr
&
Timbre: Electronic Music 1950\sphinxhyphen{}1990
\\
\hline&&
12:00\sphinxhyphen{}13:00 Uhr
&
\sphinxstyleemphasis{Lunch Break}
\\
\hline
11
&&
13:00\sphinxhyphen{}14:20 Uhr
&
Group work
\\
\hline
12
&&
14:40\sphinxhyphen{}16:00 Uhr
&
Recapitulation and conclusion
\\
\hline
\end{tabulary}
\par
\sphinxattableend\end{savenotes}


\section{Credits}
\label{\detokenize{1_orga:credits}}
Active participation in this course is compensated with 3 credit points (CPs),
\sphinxhref{https://verwaltung.uni-koeln.de/abteilung21/content/studienangebot/studiengaenge\_u\_\_abschluesse/bachelor\_\_und\_masterstudiengaenge/index\_ger.html}{equivalent to a work load of 90 hours}.
These are distributed as follows: 24 SWS (à 45 minutes) are allocated to presence in the block seminar.
Additionally, 24 SWS are dedicated to the preparation and follow\sphinxhyphen{}up of the material.
The remainder of 42 SWS goes to the reading of the relevant literature.


\section{Deliverables and learning objectives}
\label{\detokenize{1_orga:deliverables-and-learning-objectives}}
Apart from attending and following the presentations by the lecturer,
course work consists of three main parts: preparing the relevant literature (reading),
completing the assigned exercises (group work), and critically engaging with the course materials
in the form of a report written together with your group (report).

These deliverables will broaden your knowledge and understanding of current musicological research,
enhance your organizational and social skills, and help you to develop efficient work\sphinxhyphen{}load management strategies.
Finally, compiling a report will advance your communication and writing abilities.
\subsubsection*{Reading}

For each session, the relevant literature is cited in the text and provided on
\sphinxhref{https://www.ilias.uni-koeln.de/ilias/goto\_uk\_crs\_3528627.html}{ILIAS}.
Careful preparation of the reading material is required in order to be able to follow the content of the course.
Because the course will mainly talk about methods and general points of musical corpus research,
the content (and musical topic) will mainly be introduced by the literature.

I am aware that the reading workload is relatively high since the course will be taught as a block seminar
and doesn’t spread out over the entire semester. I hope that the fact that the course is finished before the
end of the year compensates for this.
\subsubsection*{Group work}

At the beginning of the course, you will be randomly assigned to a group.
Together with your group (which will stay fixed for the entire semester),
you will work on a number of exercises during the course, e.g. in Zoom breakout rooms.
You will collaborate on specific tasks related to the topic at hand, discuss methodological questions,
and help each other in the understanding of some of the concepts that are introduced in the course.
\subsubsection*{Report}

After the course has ended, your group will be randomly assigned a course topic (one of the twelve sessions in \hyperref[\detokenize{1_orga:tab-overview}]{Table \ref{\detokenize{1_orga:tab-overview}}}).
It is your task to write a report on this theme. Questions that you could address are:
What did you learn? Which concepts are not clear? Which methods did you (not) understand?
What is missing? How can the textual descriptions be improved? Who in your group did what?
Was the presentation of the material adequate? If not, what was missing?
Please also write about the organization of your group, challenges and benefits.

\sphinxstylestrong{Recommended structure for the report}
\begin{enumerate}
\sphinxsetlistlabels{\arabic}{enumi}{enumii}{}{.}%
\item {} 
\sphinxstylestrong{Introduction:} general description and summary of the course and your assigned session in particular.

\item {} 
\sphinxstylestrong{Discussion:} summarize the main discussion, open questions, and how you would continue this line or research.

\item {} 
\sphinxstylestrong{Issues:} describe in detail what was crucial for your understanding of the topic, what was missing, etc.

\item {} 
\sphinxstylestrong{Various:} anything that you would like to write in the report

\item {} 
\sphinxstylestrong{Author contributions:} describe briefly how each of you specifically contributed to the report.

\end{enumerate}

\begin{sphinxadmonition}{important}{Important:}
Submit your report by \sphinxstylestrong{31 January 2021, 23:59h} to \sphinxhref{mailto:fabian.moss@epfl.ch}{fabian.moss@epfl.ch}
as a single PDF file per group, named \sphinxtitleref{intro\_corpusmus\_\textless{}group\_number\textgreater{}.pdf}, e.g. \sphinxtitleref{intro\_corpusmus\_1.pdf}.
\end{sphinxadmonition}


\chapter{Folk Songs and the Melodic Arc}
\label{\detokenize{3_folk_songs:folk-songs-and-the-melodic-arc}}\label{\detokenize{3_folk_songs::doc}}
\noindent{\hspace*{\fill}\sphinxincludegraphics[width=1.000\linewidth]{{lines}.jpg}\hspace*{\fill}}

\clearpage

Tones are among the basic elements of music. Most musical styles combine tones in different ways
to create songs, chants, instrumental pieces, or other elaborate compositions.
In this chapter, we will analyze some basice aspects of songs by studying distributions of tones and intervals.

Huron… / MusThe Tutorial


\section{Data}
\label{\detokenize{3_folk_songs:data}}
\sphinxurl{http://kern.humdrum.org/data?f=zip\&l=/essen}
or \sphinxurl{http://kern.humdrum.org/help/data/}

Open and read \sphinxtitleref{README.txt}

Essen Folksong Collection


\section{Notes, Pitch Classes}
\label{\detokenize{3_folk_songs:notes-pitch-classes}}
\sphinxurl{https://github.com/DCMLab/DigitalMusicologyExercises/tree/master/tone\_profiles}

means, variance

also multidimensional (for later)


\section{Melodic Arc}
\label{\detokenize{3_folk_songs:melodic-arc}}
\begin{figure}[htbp]
\centering
\capstart

\noindent\sphinxincludegraphics{{melodic_arc}.png}
\caption{Melodic arc. The red lines in the background show the melodic profile of each song in the Essen Folksong Corpus;
the thick black line shows the melodic arc that was obtained by using \sphinxstyleemphasis{Locally Weighted Scatterplot Smoothing} (LOWESS) \sphinxcite{bibliography:cleveland1988}.}\label{\detokenize{3_folk_songs:id2}}\end{figure}


\section{Intervals}
\label{\detokenize{3_folk_songs:intervals}}
\sphinxurl{https://github.com/DCMLab/DigitalMusicologyExercises/tree/master/interval\_bigrams}

maybe extend with Hansen and Pearce (2014) (but data not available?)

\begin{sphinxadmonition}{note}{Note:}
In this chapter we covered the following musical terms:
\begin{itemize}
\item {} 
a

\item {} 
b

\item {} 
c

\end{itemize}
\end{sphinxadmonition}



\begin{sphinxthebibliography}{CD88}
\bibitem[CD88]{bibliography:cleveland1988}
William S Cleveland and Susan J Devlin. Locally Weighted Regression: An Approach to Regression Analysis by Local Fitting. \sphinxstyleemphasis{Journal of the American Statistical Association}, 83(403):596\textendash{}610, 1988.
\end{sphinxthebibliography}



\renewcommand{\indexname}{Index}
\printindex
\end{document}