\documentclass[aspectratio=169,
  % for xcolor:
  usenames,dvipsnames,table
]{beamer}

\renewcommand{\thesection}{\Roman{section}}
\renewcommand{\thesubsection}{\thesection.\Roman{subsection}}

\definecolor{cologneblue}{RGB}{69,122,147}

\setbeamercolor{itemize item}{fg=cologneblue}
\setbeamercolor{enumerate item}{fg=cologneblue}
\setbeamerfont{enumerate item}{series=\bfseries}

%%% PACKAGES %%%

% use the metropolis theme
\usetheme[
  progressbar=head,
  subsectionpage=progressbar,
  numbering=fraction
  ]{metropolis}
% see metropolis package for more options
% dark theme must be set after the metroepfl package is loaded (see below)

\usepackage{amsmath}
\usepackage[utf8]{inputenc}
% \usepackage[T1]{fontenc}
% \usepackage[czech]{babel}
% \usepackage{lmodern}
\usepackage{xcolor}

% use the epfl color theme
% defines epfl{gray,darkgray,lightgray,red,blue}
% this usepackage is optional, as it is also done by metroepfl
\usepackage{epflcolors}
% metropolis theme based on gemini and epflcolors
\usepackage{metroepfl}
% \metroset{background=dark}

\usepackage{multimedia} % for audio/video

\usepackage{tikz}
\usetikzlibrary{arrows,
                arrows.meta,
                calc,
                cd,
                positioning,
                backgrounds,
                graphs,
                shapes.geometric,
                fit,
                decorations,
                decorations.markings,
                decorations.pathreplacing,
                overlay-beamer-styles}
\pgfdeclarelayer{bg}    % declare background layer
\pgfsetlayers{bg,main}  % set the order of the layers (main is the standard layer)

\usepackage{ifthen}
\usepackage{amssymb,amsmath}
\let\emptyset\varnothing

\usepackage{hyperref}

%% add to show notes
\usepackage{pgfpages}
% \setbeameroption{show notes on second screen}
%% only for handout printing!
% \pgfpagesuselayout{4 on 1}[a4paper,border shrink=5mm,landscape]

\usepackage{caption}
\usepackage{subcaption}

%% References
\usepackage[
   backend=biber,
   style=apa,
   citestyle=authoryear-comp,
   % uniquename=false,
	 % uniquelist=false,
	 maxcitenames=4,
   natbib=true
 ]{biblatex}
\addbibresource{references.bib}
\setbeamertemplate{bibliography item}{} % remove icon in front of refs

\usepackage{appendixnumberbeamer} % slide numbers in appendix

%%% FRONTMATTER %%%

\title{Introduction to Musical Corpus Studies}
% \subtitle{WS 2020/21}
\author{Fabian C. Moss}
\date{13 November 2020}
\institute{Musikwissenschaftliches Seminar - Universität zu Köln - WS 2020/21}
\titlegraphic{\vspace{6cm}\flushright\includegraphics[width=3cm,height=2cm, keepaspectratio]{img/uzk_logo.jpg}}

%% change appearance of progress bar
\makeatletter
\setlength{\metropolis@progressinheadfoot@linewidth}{3pt}
\setlength{\metropolis@titleseparator@linewidth}{3pt}
\setlength{\metropolis@progressonsectionpage@linewidth}{3pt}

\begin{document}

\maketitle

%%%% MAIN MATTER %%%%%%%%%%%%%%%%%%%%%%%%%%%%%%%%%%%%%%%%%


\begin{frame}{Outline}
  \begin{enumerate}[I.]
    \item What are Musical Corpus Studies?
    \item Issues
    \item Examples
    \item Organization of the course
    \item Questions
  \end{enumerate}
\end{frame}

\begin{frame}{Resources}
  \begin{itemize}
    \item main organization via ILIAS
    \item literature
    \item forum
    \item Zoom link (you are all here)
    \item external website: \url{https://fabianmoss.github.io/intro-corpusmus}
    \begin{itemize}
      \item general info
      \item course materials (updated after each session)
    \end{itemize}
    \item HfMT students: by your group members
  \end{itemize}
\end{frame}

\begin{frame}{Credit Points}
  \begin{itemize}
    \item 3 CPs = 90 SWS
    \begin{itemize}
      \item 24 SWS presence in seminar
      \item 24 SWS preparation of and follow-up on course materials
      \item 42 SWS reading of literature and writing of report
    \end{itemize}
  \end{itemize}
\end{frame}

\begin{frame}{Group work}
  \begin{itemize}
    \item you will meet with your group in the breakout rooms
    \item discussions
    \item exercises
  \end{itemize}
\end{frame}

\begin{frame}{Report}
  \begin{itemize}
    \item report due on \textbf{31 January 2021, 23:59h}
    \item 6--8 pages
    \item suggested structure
    \begin{enumerate}
      \item Introduction
      \item Discussion
      \item Issues
      \item Various
      \item Contributions
    \end{enumerate}
  \end{itemize}
\end{frame}
\section{\thesection.~What are Musical Corpus Studies}
\input{2_issues.tex}
\section{\thesection.~Examples}

\begin{frame}{Example}
    Example of our most recent research:
    \begin{itemize}
        \item \fullcite{Harasim2020}
    \end{itemize}
\end{frame}

\begin{frame}{Research questions}
    \begin{enumerate}
        \item How can we find modes \alert{automatically}? 
        \item How can the concept of a \alert{mode} be operationalized? 
        \item Can we do it without knowing (\alert{unsupervised}) how many modes there are and what they look like?
        \item How do modes change \alert{historically}?
    \end{enumerate}
\end{frame}

\begin{frame}{Corpus}
    \begin{itemize}
        \item 21'000 pieces from \url{https://classicalarchives.com}
        \item MIDI format
        \item user-generated (quality?)
        \item biases
        \item metadata: composer names, keys, composition date, \ldots 
        \item representativeness?
        \item almost no early music examples $\longrightarrow$ add from other projects
        \begin{enumerate}
            \item \href{http://crimproject.org/}{\emph{Citations: The Renaissance Imitation Mass Project} (CRIM)}
            \item \href{http://digitalduchemin.org/}{\emph{The Lost Voices Project}}
        \end{enumerate}
    \end{itemize}
    \pause
    $\Longrightarrow$ in total 13'402 pieces (ca. 55 million notes) with given composition year (but not key)
\end{frame}

\begin{frame}{Corpus}
    \begin{figure}
        \centering
        \includegraphics[width=\linewidth,height=.8\textheight,keepaspectratio]{img/Figure1.pdf}
        \caption{Historical distribution of pieces in the corpus.}
        \label{fig:piece_dist}
    \end{figure}
\end{frame}

\begin{frame}{Assumptions}
    \begin{enumerate}
        \item pieces can be represented by pitch-class counts
        \item enharmonic equivalence
        \item transpositional invariance
    \end{enumerate}

    \pause
    
    All of these assumptions are highly questionable, especially on a large historical scale!
    
    \pause
    
    $\Longrightarrow$ explitic modeling
\end{frame}

\begin{frame}{An example}
    \begin{figure}
        \centering
        \includegraphics[width=.6\linewidth]{img/piano_pitch_class.eps}
        \caption{Pitch-class counts of an example piece in C~major.}
    \end{figure}
\end{frame}

\begin{frame}{The model}
    
\end{frame}

\begin{frame}{Automatically finding modes}
    \begin{figure}
        \centering
        \includegraphics[width=\linewidth,height=.8\textheight,keepaspectratio]{img/Figure4.pdf}
        \caption{Three models for automatic mode finding.}
        % \label{fig:piece_dist}
    \end{figure}
\end{frame}

\begin{frame}{Quality of the model}
    \begin{figure}
        \centering
        \includegraphics[width=\linewidth,height=.7\textheight,keepaspectratio]{img/Figure5.pdf}
        \caption{Accuracy scores of our model in five historical periods.}
        % \label{fig:piece_dist}
    \end{figure}
\end{frame}

\begin{frame}{The major and minor modes}

    Pitch-class distributions of all pieces in the Baroque and Classical periods:

    \begin{figure}
        \centering
        \includegraphics[width=\linewidth,height=.7\textheight,keepaspectratio]{img/Figure8.pdf}
        \caption{Pitch-class distribution of the major and minor modes.}
        % \label{fig:piece_dist}
    \end{figure}
\end{frame}

\begin{frame}{Modes in the Renaissance}
    \begin{figure}
        \centering
        \includegraphics[width=\linewidth,height=.7\textheight,keepaspectratio]{img/Figure6.pdf}
        \caption{Clustering into four modes in the Renaissance.}
        % \label{fig:piece_dist}
    \end{figure}
\end{frame}

\begin{frame}{Modes in the Renaissance}
    \begin{figure}
        \centering
        \includegraphics[width=.6\linewidth]{img/renaissance_modes.png}
        \caption{Six modes in early music.}
    \end{figure}
\end{frame}

\begin{frame}{Modes in the Renaissance}
    \begin{columns}
        \begin{column}{.6\linewidth}
            \begin{figure}
                \centering
                \includegraphics[width=\linewidth,height=.8\textheight,keepaspectratio]{img/Figure7.pdf}
                \caption{Pitch-class distribution of Renaissance modes.}
                % \label{fig:piece_dist}
            \end{figure}
        \end{column}
        \hfill
        \begin{column}{.4\linewidth}
            Four modes emerge in the Renaissance
            
            \begin{itemize}
                \item Mixolydian (violet)
                \item Ionian (red)
                \item Dorian (blue)
                \item Aeolian/Dorian (green)
            \end{itemize}
        \end{column}
    \end{columns}
\end{frame}

\begin{frame}{Summary}
    \begin{enumerate}
        \item \ldots
    \end{enumerate}
\end{frame}
\section{\thesection.~Course organization}

\begin{frame}{Resources}
    \begin{itemize}
      \item main organization via ILIAS
      \item literature
      \item forum
      \item Zoom link (you are all here)
      \item external website: \url{https://fabianmoss.github.io/intro-corpusmus}
      \begin{itemize}
        \item general info
        \item course materials (updated after each session)
      \end{itemize}
      \item HfMT students: by your group members
  \end{itemize}
\end{frame}

\begin{frame}{Credit Points}
  \begin{itemize}
    \item 3 CPs = 90 SWS
    \begin{itemize}
      \item 24 SWS presence in seminar
      \item 24 SWS preparation of and follow-up on course materials
      \item 42 SWS reading of literature and writing of report
    \end{itemize}
  \end{itemize}
\end{frame}

\begin{frame}{Group work}
  \begin{itemize}
    \item you will meet with your group in the breakout rooms
    \item discussions
    \item exercises
    \item Let's test the breakout rooms! (5--7 min for contact info exchange)
  \end{itemize}
\end{frame}

\begin{frame}{Report}
  \begin{itemize}
    \item report due on \textbf{31 January 2021, 23:59h}
    \item 6--8 pages
    \item suggested structure
    \begin{enumerate}
      \item Introduction
      \item Discussion
      \item Issues
      \item Various
      \item Contributions
    \end{enumerate}
  \end{itemize}
\end{frame}

\begin{frame}
  \begin{center}
    \alert{\textbf{Questions?}}
  \end{center}
\end{frame}
% \section{\thesection.~Understanding Chord Distributions}

% \againframe<5-6>{disciplines}

\begin{frame}{\insertsectionhead}
  \begin{enumerate}[\color{epflred}1.]
    \item<1-> \textbf{Corpus creation}\\ \citet{Neuwirth2018}.~\citetitle{Neuwirth2018}.
    \item<2-> \textbf{Corpus analysis}\\\citet{Moss2019b}.~\citetitle{Moss2019b}.
    \item<3-> \textbf{Corpus expansion}\\ \citet{Moss2019a}.~\citetitle{Moss2019a} (Diss.).
    % \item \emph{Distant Listening} project (2020--2024)
    % \item Mozart Sonatas (in preparation)
    % \item 24 other composers in preparation (16th--20th c.)
  \end{enumerate}
\end{frame}

\begin{frame}{\insertsectionhead}
  19th-century corpora (9 composers, 289 pieces)
  \begin{columns}
    \begin{column}{.4\linewidth}
      \begin{itemize}
        \item Beethoven: String quartets
        \item Schubert: \emph{Winterreise}
        \item Chopin: Mazurkas
        \item Liszt: \emph{Années de pèlerinage}
        \item Dvorák: \emph{Silhouettes}
        \item Grieg: \emph{Lyrical pieces}
        \item Tchaikovsky: \emph{Seasons}
        \item Debussy: \emph{Suite bergamasque}
        \item Medtner: \emph{Fairy tales}
      \end{itemize}
    \end{column}
    \pause
    \begin{column}{.6\linewidth}
      \begin{figure}
        \centering
        \includegraphics[width=\textwidth,height=.8\textheight,keepaspectratio]{img/musescore_screenshot.png}
        \caption{Chord annotation via \emph{MuseScore}.}
        \label{}
      \end{figure}
    \end{column}
  \end{columns}


\end{frame}

\begin{frame}<1-7>[label=corpus_pipeline]{\insertsectionhead}
  \onslide<1->{The corpus research pipeline}

  \resizebox{\textwidth}{!}{
  \centering
  \tikzstyle{my_arrow} = [->,>=stealth,shorten >= .5em, shorten <= .5em, line width=2pt]
  \begin{tikzpicture}
    \draw [visible on=<2->] (0,0) node (score) {\includegraphics[height=1.5cm]{img/score.png}};

    \draw [visible on=<3->] (3,0) node (file) {\includegraphics[height=1.5cm]{img/xml.png}};
    \draw [my_arrow,visible on=<3->] (score) -- (file) node [above,midway] {encoding};

    \draw [visible on=<4->] (6.5,3) node (expert) {\includegraphics[height=1.5cm]{img/expert.png}};
    \draw [my_arrow,epflred,visible on=<4-7>] (expert) -- (file) node [midway,above left] {\color{epfldark}{analysis}};
    \draw [my_arrow,epflred!30,visible on=<8->] (expert) -- (file) node [midway,above left] {\color{epfldark!30}{analysis}};

    \draw [visible on=<5-7>] (7,0) node (vec) {\footnotesize$x=\begin{bmatrix}I\\V7\\ii6\\V7/IV\\\sharp viio\\\vdots\end{bmatrix}$};
    \draw [visible on=<8->] (7,0) node (vec) {\footnotesize$x=\begin{bmatrix}B\flat\\G\\A\\G\\C\sharp\\\vdots\end{bmatrix}$};
    \draw [my_arrow,visible on=<5->] (file) -- (vec) node [above,midway] {extraction};

    \draw [visible on=<6-7>] (12,0) node (dist) {\includegraphics[width=4cm]{img/chord_stats.pdf}};
    \draw [visible on=<8->] (12,0) node (dist) {\includegraphics[width=4cm]{img/tpc_dists_sorted.pdf}};
    \draw [my_arrow,visible on=<6->] (vec) -- (dist) node [above, midway] {statistics};

    \draw [my_arrow,visible on=<7->, epflred] (expert) -- (dist) node [above right, midway] {\color{epfldark}{interpretation}};

  \end{tikzpicture}
  }
\end{frame}

\begin{frame}{\insertsectionhead}
  A fundamental distinction in corpus research:

  \begin{itemize}
    \item \alert{all chords} that occur in a corpus $\mathcal C$ (chord tokens), e.g.
  \end{itemize}
  \begin{align*}
    T_{\mathcal C} = \{\, I : 12,\, V7 : 6,\, V : 5,\, ii : 5,\, IV : 4,\, I6 : 2,\, III : 1,\, viio7 : 1\,\}
  \end{align*}
  \pause
  \begin{itemize}
    \item \alert{all unique chords} that occur in a corpus $\mathcal C$ (chord types), e.g.
  \end{itemize}
  \begin{align*}
    t_{\mathcal C} = \{\, I, V7, V, ii, IV, I6, III, viio7 \,\}
  \end{align*}
\end{frame}


\begin{frame}{\insertsectionhead}

  \begin{columns}
    \begin{column}{.7\textwidth}
\def\firstcircle{(0,0) circle (1.2cm)}
\def\secondcircle{(60:2cm) circle (2cm)}
\def\thirdcircle{(120:2cm) circle (1.5cm)}
\tikzstyle{corpus}=[line width=0mm,epflred,fill, opacity=0]
\begin{center}
\begin{tikzpicture}
  \draw[corpus,fill opacity=0.2, visible on=<1->]
    \firstcircle node[black,below,opacity=1] {$\mathcal C_1$};
    %
    \draw[corpus,fill opacity=0.4, visible on=<2->]
    \secondcircle node[black,right,opacity=1] {$\mathcal C_2$};
    %
    \draw[corpus,fill opacity=0.6, visible on=<3->]
    \thirdcircle node[black,left,opacity=1] {$\mathcal C_3$};
    %
    \draw [thick,decorate,decoration={brace,amplitude=10pt},visible on=<4->] (-3,-1.5) -- (-3,4) node [midway, left, xshift=-5mm,align=center]  {Vocabulary:\\3,185 unique chords\\(75,380 chords in total)}; % \\3185 types
    \node [align=center, visible on=<5->] (core) at (3,-1) {Core:\\43 unique chords}; %  $\mathbb C$\\43 types
    \node (intersection) at (-.45,.85) {};
    \draw[->,>=stealth,thick, visible on=<5->] (core) -- (intersection);
\end{tikzpicture}
\end{center}
\end{column}
%
\begin{column}{.3\textwidth}
  \onslide<6->{
\begin{align*}
  \frac{|Core|}{|Vocabulary|} \approx 1.35\text{\%}
\end{align*}}
\end{column}
\end{columns}
\end{frame}

\begin{frame}{\insertsectionhead}
\onslide<1->{
\small
  \begin{align*}
   \text{Core} = \{\;
      & \mathsf{I},\, \mathsf{I6},\, \mathsf{I64},\, \mathsf{i},\, \mathsf{i6},\, \mathsf{i64},\, \nonumber\\
      & \mathsf{ii},\, \mathsf{ii6},\, \mathsf{ii7},\, \mathsf{ii65},\, \mathsf{iio},\, \mathsf{ii\emptyset 7},\, \mathsf{ii\emptyset 43},\, \mathsf{ii\emptyset 2},\, \nonumber\\
      & \mathsf{iii},\, \mathsf{iii6},\, \mathsf{III},\, \nonumber\\
      & \mathsf{IV},\, \mathsf{IV6},\, \mathsf{iv},\, \mathsf{iv64},\, \nonumber\\
      & \mathsf{V},\, \mathsf{V6},\, \mathsf{V64},\, \mathsf{V7},\, \mathsf{V7(4)},\, \mathsf{V65},\, \mathsf{V43},\, \mathsf{V2},\, \nonumber\\
      & \mathsf{vi},\, \mathsf{vi6},\, \mathsf{VI},\, \mathsf{VI6},\, \nonumber\\
      & \mathsf{viio6},\, \mathsf{\sharp viio6},\, \mathsf{\sharp viio43},\, \nonumber\\
      & \mathsf{V65/III},\, \mathsf{V7/IV},\, \mathsf{V2/IV},\, \mathsf{V65/V},\, \mathsf{V43/V},\, \mathsf{V2/V},\, \mathsf{\sharp viio7/vi} \;\}
      \label{eq:core}
  \end{align*}}

  \onslide<2->{
  \normalsize
  Between 45 and 74\% of all chords are core chords!
  }
\end{frame}
%
\note[itemize]{

\item this collection consists of the \alert{typical cases} one would find in a harmony text book
\item consequently, textbook harmonic analysis takes only ~1\% of all chords into account
\item How can this be justified?
}

\begin{frame}{\insertsectionhead}
  $\Rightarrow$ What affects the frequency of chords?
\end{frame}

\begin{frame}{\insertsectionhead}
  \begin{columns}
    \begin{column}{.53\linewidth}
      \textbf{Zipf's law}~\citep{Zipf1949}

      The relation between \alert{frequency}~$f$ and \alert{rank}~$r$ (1st, 2nd,\ldots, $n$th) resembles a power law:
      \begin{align*}
        f(r) \approx \frac{\alpha}{(\beta + r)^\gamma}
      \end{align*}
    \end{column}\hfill
    \pause
    \begin{column}{.47\linewidth}
      \begin{figure}
        \centering
        \includegraphics[width=\textwidth]{img/zipf_model.png}
      % \caption{Zipf curve with arbitrary parameters}
      \end{figure}
    \end{column}
  \end{columns}
  % This relation is ubiquitous in linguistics and musicology \parencite{Rohrmeier2008,Piantadosi2014}. Do the annotated corpora \alert{conform} to this phenomenon?
\end{frame}

\begin{frame}{\insertsectionhead}
  \begin{figure}
    % \begin{subfigure}[c]{.6\linewidth}
      \centering
      \includegraphics[width=\textwidth]{img/ABC_zipf_mandelbrot.pdf}
      \caption{Chord distribution in Beethoven's string quartets.}
    % \end{subfigure}\vfill
    % \begin{subfigure}[c]{.6\linewidth}
    %   \centering
    %   \includegraphics[width=\textwidth]{img/debussy_zipf_mandelbrot.pdf}
    %   \caption{Debussy's \emph{Suite bergamasque}.}
    % \end{subfigure}
  \end{figure}
\end{frame}

\begin{frame}{\insertsectionhead}
  \begin{enumerate}[\color{epflred}1.]
    \item<1-> Zipf's law is found in many domains (e.g. language, social networks, citations, \ldots)
    \item<2-> What is specific about tonal music?
    \item<3-> Many processes could explain this finding~\citep{Piantadosi2014}
    % \item<4-> More sophisticated models are needed
  \end{enumerate}
  \onslide<4->{$\Rightarrow$ Addressing these issues by closer inspecting \alert{deviations} from the model}
\end{frame}

\begin{frame}{\insertsectionhead}
  \begin{figure}
    \centering
    \includegraphics[width=.8\textwidth]{img/freqrank_disterror.pdf}
    \caption{Zipf-fit of chord distribution (left). Deviations from model (right).}
    \label{}
  \end{figure}
\end{frame}

\begin{frame}{\insertsectionhead}
  \begin{itemize}
    \item<1-> \alert{new insight}: systematic deviations from standard models of chord distributions
    \item<2-> development of more advanced models in future research
    \item<3-> impossible without the computational approach
  \end{itemize}

\end{frame}

% \begin{frame}{Summary and Prospects}
%   \begin{enumerate}[\color{epflred}1.]
%     \item \ldots
%     \item \ldots
%     \item \ldots
%     \label{}
%   \end{enumerate}
% \end{frame}

% \section{\thesection.~A Data-Driven History of Tonality}

% \againframe<{5,7}>{disciplines}

% \begin{frame}{\insertsectionhead}
%   Motivation / Questions
% \end{frame}

\againframe<7->{corpus_pipeline}

\subsection{1. Recovering the line of fifths}

\begin{frame}{\insertsectionhead}
    \onslide<3->{
    \begin{center}
    \resizebox{.75\textwidth}{!}{
      % colors for line of fifths
\definecolor{Gb}{HTML}{6d6dff}
\definecolor{Db}{HTML}{8585ff}
\definecolor{Ab}{HTML}{9d9dff}
\definecolor{Eb}{HTML}{b5b5ff}
\definecolor{Bb}{HTML}{cdcdff}
\definecolor{F}{HTML}{e5e5ff}
\definecolor{C}{HTML}{fffdfd}
\definecolor{G}{HTML}{ffe5e5}
\definecolor{D}{HTML}{ffcdcd}
\definecolor{A}{HTML}{ffb5b5}
\definecolor{E}{HTML}{ff9d9d}
\definecolor{B}{HTML}{ff8585}
\definecolor{Fs}{HTML}{ff6d6d}

% draw picture
\begin{tikzpicture}[scale=2]

\draw[latex-latex] (-3.5,0) -- (3.5,0) ; % line

% down ticks
\foreach \x/\label in  {-6/G$\flat$,
                         -5/D$\flat$,
                         -4/A$\flat$,
                         -3/E$\flat$,
                         -2/B$\flat$,
                         -1/F,
                         0/C,
                         1/G,
                         2/D,
                         3/A,
                         4/E,
                         5/B,
                         6/F$\sharp$}
\draw[shift={(\x/2,0)},color=black] (0pt,0pt) -- (0pt,-5pt) node[below] {\label};
% % up ticks
% \foreach \x in  {-6,...,6}
% \draw[shift={(\x/2,0)},color=black] (0pt,0pt) -- (0pt,5pt) node[above] {$\x$};

% draw circles
\foreach \x/\color in {-6/Gb,
                         -5/Db,
                         -4/Ab,
                         -3/Eb,
                         -2/Bb,
                         -1/F,
                         0/C,
                         1/G,
                         2/D,
                         3/A,
                         4/E,
                         5/B,
                         6/Fs}
  \node [circle, fill=\color,scale=1.5, draw=black] (\x) at (\x/2,0) {};

\end{tikzpicture}
}
    \end{center}
  }

  \begin{figure}
    \onslide<2->{
    \begin{subfigure}[c]{.48\linewidth}
      \centering
      \includegraphics[width=\textwidth]{img/tpc_dists_sorted.pdf}
      \subcaption{Ranked note frequencies.}
    \end{subfigure}
    }
    \hfill
    \onslide<4->{
    \begin{subfigure}[c]{.48\linewidth}
      \centering
      \includegraphics[width=\textwidth]{img/tpc_dists.pdf}
      \subcaption{Note frequencies on line of fifths.}
    \end{subfigure}
    }
    \caption{F. Schubert, \emph{Impromptu}, op.~90/2 in E$\flat$~major.}
    \label{}
  \end{figure}
\end{frame}

\note[itemize]{
  \item Using Music Theory: arranging on LoF shows structure
  \item define fifth width
  \item FW >= 7: chromaticism; FW >= 12: enharmonicism
}

\begin{frame}{\insertsectionhead}

  Corpus studies allow us to
  \begin{itemize}
    \item widen the \alert{scope} to entire collections of pieces
    \item empirically \alert{test} the validity of music-theoretical concepts
  \end{itemize}

  \pause

  $\Rightarrow$ Is the line of fifths \emph{really} a \alert{relevant structure} for pitch-class distributions?

  \pause

  \begin{enumerate}[\color{epflred}1.]
    \item Gather a large corpus.
    \item Transform pieces to pitch-class distributions.
    \item Apply \emph{dimensionality reduction} to uncover underlying structures.
  \end{enumerate}
\end{frame}

\begin{frame}{\insertsectionhead}

  \begin{columns}
    \begin{column}{.3\textwidth}
      % Sources
      % \begin{itemize}
      %   \item existing published \& accessible data
      %   \item web search
      %   \item transcription
      % \end{itemize}
      % Dataset \citep{Moss2020}
      The corpus:
      \begin{itemize}
        \item 2,012 pieces
        \item 75 composers
        \item approx. 600 years
      \end{itemize}
    \end{column}
    %
    \begin{column}{.7\textwidth}
      \begin{figure}
        \centering
        \includegraphics[width=\textwidth]{img/piece_dist.pdf}
        \caption{Historical distribution of the corpus.}
        \label{}
      \end{figure}
    \end{column}
  \end{columns}
\end{frame}

\begin{frame}{\insertsectionhead}
  \begin{figure}
  \captionsetup[subfigure]{justification=centering}
    \onslide<2->{
      \begin{subfigure}[t]{.3\textwidth}
        \centering
        \includegraphics[width=\linewidth]{img/dim_reduct_Isomap.pdf}
        \subcaption*{Isomap.}
      \end{subfigure}
    }
    \hfill
    \onslide<1->{
      \begin{subfigure}[t]{.3\textwidth}
        \centering
        \includegraphics[width=\linewidth]{img/dim_reduct_PCA.pdf}
        \subcaption*{Principal Component Analysis.}
      \end{subfigure}
    }
    \hfill
    \onslide<2->{
    \begin{subfigure}[t]{.3\textwidth}
      \centering
      \includegraphics[width=\linewidth]{img/dim_reduct_MDS.pdf}
      \subcaption*{Multi-Dimensional Scaling.}
    \end{subfigure}
    }
    \caption{Dimensionality reduction of tonal space.}
    \label{}
  \end{figure}

  \onslide<1->{
    \begin{center}
    \resizebox{.75\textwidth}{!}{
      % colors for line of fifths
\definecolor{Gb}{HTML}{6d6dff}
\definecolor{Db}{HTML}{8585ff}
\definecolor{Ab}{HTML}{9d9dff}
\definecolor{Eb}{HTML}{b5b5ff}
\definecolor{Bb}{HTML}{cdcdff}
\definecolor{F}{HTML}{e5e5ff}
\definecolor{C}{HTML}{fffdfd}
\definecolor{G}{HTML}{ffe5e5}
\definecolor{D}{HTML}{ffcdcd}
\definecolor{A}{HTML}{ffb5b5}
\definecolor{E}{HTML}{ff9d9d}
\definecolor{B}{HTML}{ff8585}
\definecolor{Fs}{HTML}{ff6d6d}

% draw picture
\begin{tikzpicture}[scale=2]

\draw[latex-latex] (-3.5,0) -- (3.5,0) ; % line

% down ticks
\foreach \x/\label in  {-6/G$\flat$,
                         -5/D$\flat$,
                         -4/A$\flat$,
                         -3/E$\flat$,
                         -2/B$\flat$,
                         -1/F,
                         0/C,
                         1/G,
                         2/D,
                         3/A,
                         4/E,
                         5/B,
                         6/F$\sharp$}
\draw[shift={(\x/2,0)},color=black] (0pt,0pt) -- (0pt,-5pt) node[below] {\label};
% % up ticks
% \foreach \x in  {-6,...,6}
% \draw[shift={(\x/2,0)},color=black] (0pt,0pt) -- (0pt,5pt) node[above] {$\x$};

% draw circles
\foreach \x/\color in {-6/Gb,
                         -5/Db,
                         -4/Ab,
                         -3/Eb,
                         -2/Bb,
                         -1/F,
                         0/C,
                         1/G,
                         2/D,
                         3/A,
                         4/E,
                         5/B,
                         6/Fs}
  \node [circle, fill=\color,scale=1.5, draw=black] (\x) at (\x/2,0) {};

\end{tikzpicture}
}
    \end{center}
  }

\end{frame}

\note[itemize]{
  \item Uncovering underlying spaces: dimensionality reduction
  \item Empirical validation of MT conception of tonal space
  \item comparison of different methods shows that result is robust
}

\begin{frame}{\insertsectionhead}

  \begin{enumerate}[\color{epflred}1.]
    \item<1-> \alert{Historical changes} in the distributions of pitch classes?
    \item<2-> Important \alert{transitions} in the way composers write music on a large scale?
    \item<3-> Observe the \alert{expansion} of tonal material on line of fifths.
  \end{enumerate}

\end{frame}

\begin{frame}{\insertsectionhead}
  \begin{figure}
    \centering
    \includegraphics[width=\textwidth]{img/fifth_width.pdf}
    \caption{Historical expansion of tonal material on line of fifths.}
    \label{}
  \end{figure}
\end{frame}

\note[itemize]{
\item Dufay: F\#\# instead of g
\item Ockeghem: Cb instead of C\#
}

\subsection{2. Modelling pieces on the \emph{Tonnetz}}

\begin{frame}{\insertsectionhead}

  Line of fifths is fundamental and simple model but does not explain everything.
  \pause

  $\Rightarrow$ Extend approach to more advanced \alert{models of tonal space}

  \pause
  % Modeling of tonal space in the 19th century: Hauptmann, v. Oettingen, Hostinský, Riemann

  \begin{figure}
    \centering
    \includegraphics[width=.75\textwidth]{img/hostinsky_tonnetz.png}
    \caption{The \emph{Tonnetz} \citep{Hostinsky1879}.}
    \label{}
  \end{figure}

\end{frame}

\begin{frame}[c]{\insertsectionhead}
  Neo-Riemannian \alert{triadic transformations} on the \emph{Tonnetz}~\citep{Cohn1998}
  \begin{columns}
    \begin{column}{.45\textwidth}
      \begin{itemize}
        \item<2-> \textcolor{epfldark}{parallel ($\mathbf{P}$)}\\ \includegraphics{scores/parallel.pdf}
        \item<3-> \textcolor{canard}{relative ($\mathbf{R}$)}\\ \includegraphics{scores/relative.pdf}
        \item<4-> \textcolor{leman}{leading-tone exchange ($\mathbf{L}$)}\\ \includegraphics{scores/leading_tone.pdf}
      \end{itemize}

    \end{column}
    %
    \begin{column}{.65\textwidth}
      % TONNETZ WITH ROBERT'S CODE
      %% Figure adapted from https://github.com/DCMLab/TikZ_plot_templates/blob/master/example.tex

  \begin{figure}[c]
    \centering
    % adjust size
    \def\minx{-2}
    \def\maxx{5}
    \def\miny{-2}
    \def\maxy{4}

    % draw hexagons (comment out for not plotting)
    \def\withhex{}
    \newcommand*{\hex}[1]{
      \def\r{0.55}
      \draw[black!7,fill=black!6,visible on=<5->] ($(#1)+(30:\r)$) -- ($(#1)+(90:\r)$) -- ($(#1)+(150:\r)$)  -- ($(#1)+(210:\r)$) -- ($(#1)+(270:\r)$) -- ($(#1)+(330:\r)$) -- cycle;
    }

    \newcommand*{\xy}[2]{
      \pgfmathsetmacro{\x}{#1+cos(60)*#2}
      \pgfmathsetmacro{\y}{sin(60)*#2}
    }
    \newcommand*{\xycoord}[2]{($#1*(1,0)+#2*(60:1)$)}
    \newcommand*{\xycoordcenterbelow}[2]{(${#1+0.333}*(1,0)+{#2+0.333-1}*(60:1)$)}
    \newcommand*{\xycoordcenterabove}[2]{(${#1-0.333}*(1,0)+{#2+0.666}*(60:1)$)}

    \def\r{0.55}
    \tikzstyle{tone}=[circle, draw, fill=white, minimum size=2.5em, thick]
    \scalebox{0.6}{
    \begin{tikzpicture}[scale=2, every path/.style={thick}]
      % create the basic grid
      \foreach \x in {-10,...,10} {
        \foreach \y in {-10,...,10} {
          \pgfmathsetmacro{\xx}{\x+cos(60)*mod(\y+100,2)}
          \pgfmathsetmacro{\yy}{sin(60)*\y}
          \ifthenelse{\x>\minx\AND\x<\maxx\AND\y>\miny\AND\y<\maxy}{
            \ifdefined\withhex\hex{\xx,\yy}\fi
            \coordinate (n_\x_\y) at (\xx,\yy);
            \pgfmathsetmacro{\xMOne}{int(\x-1)}
            \pgfmathsetmacro{\yMOne}{int(\y-1)}
            \pgfmathsetmacro{\yPOne}{int(\y+1)}
            \pgfmathsetmacro{\xxMOne}{\xMOne+cos(60)*mod(\yMOne+100,2)}
            \pgfmathsetmacro{\yyMOne}{sin(60)*\yMOne}
            \pgfmathsetmacro{\yyPOne}{sin(60)*\yPOne}
            \pgfmathsetmacro{\iseven}{int(mod(\y+100,2))}
            \ifthenelse{\xMOne>\minx\AND\xMOne<\maxx}{
              \draw (n_\x_\y) -- (n_\xMOne_\y);
            }{}
            \ifthenelse{\yMOne>\miny\AND\yMOne<\maxy}{
              \draw (n_\x_\y) -- (n_\x_\yMOne);
            }{}
            \ifthenelse{\xMOne>\minx\AND\xMOne<\maxx\AND\yMOne>\miny\AND\yMOne<\maxy}{
              \ifthenelse{\iseven=0}{
                \draw (n_\x_\y) -- (n_\xMOne_\yMOne);
              }{}
            }{}
            \ifthenelse{\xMOne>\minx\AND\xMOne<\maxx\AND\yPOne>\miny\AND\yPOne<\maxy}{
              \ifthenelse{\iseven=0}{
                \draw (n_\x_\y) -- (n_\xMOne_\yPOne);
              }{}
            }{}
          }{}
        }
      }
      % coloured areas
      % \draw[fill=epflred, opacity=0.3] \xycoord{0}{0} -- \xycoord{2}{0} -- \xycoord{2}{1} -- \xycoord{0}{1} -- cycle;
      \draw[fill=epflred, opacity=0.4,visible on=<1->] \xycoord{0}{0} -- \xycoord{1}{0} -- \xycoord{0}{1} -- cycle; % C major
      \draw[fill=epflred, opacity=0.2,visible on=<2->] \xycoord{0}{0} -- \xycoord{1}{0} -- \xycoord{1}{-1} -- cycle; % C minor
      \draw[fill=epflred, opacity=0.2,visible on=<3->] \xycoord{0}{0} -- \xycoord{0}{1} -- \xycoord{-1}{1} -- cycle; % A minor
      \draw[fill=epflred, opacity=0.2,visible on=<4->] \xycoord{1}{0} -- \xycoord{1}{1} -- \xycoord{0}{1} -- cycle; % E minor


      % create tone names
      \foreach \nodename\name/\fifths/\thirds in {
        A+/{A$\sharp$}/-2/3, E+/{E$\sharp$}/-1/3, B+/{B$\sharp$}/0/3, F++/{F$\sharp\sharp$}/1/3, C++/{C$\sharp\sharp$}/2/3, G++/{G$\sharp\sharp$}/3/3,
        F+/{F$\sharp$}/-2/2, C+/{C$\sharp$}/-1/2, G+/{G$\sharp$}/0/2, D+/{D$\sharp$}/1/2, A+/{A$\sharp$}/2/2, E+/{E$\sharp$}/3/2,
        A/A/-1/1, E/E/0/1, B/B/1/1, F+/{F$\sharp$}/2/1, C+/{C$\sharp$}/3/1, G+/{G$\sharp$}/4/1,
        F/F/-1/0, C/C/0/0, G/G/1/0, D/D/2/0, A/A/3/0, E/E/4/0,
        A-/{A$\flat$}/0/-1, E-/{E$\flat$}/1/-1, B-/{B$\flat$}/2/-1, F/F/3/-1, C/C/4/-1, G/G/5/-1
      } {
        \node[tone] (\nodename) at \xycoord{\fifths}{\thirds} {\name};
      }
      % draw some arrows
      % \draw[->, >=stealth, shorten >= 1.5em, shorten <= 1.5em, line width=3pt, epflred] \xycoord{0}{0} -- \xycoord{1}{1};
      % \draw[->, >=stealth, line width=3pt, violet] \xycoordcenterbelow{1}{1} -- \xycoordcenterabove{3}{0};
      % \draw[->, >=stealth, line width=3pt, green] \xycoord{0.5}{1.5} -- \xycoord{1.5}{1.5};
      % \draw[->, >=stealth, line width=3pt, blue] \xycoord{-0.75}{2.5} -- \xycoord{1.25}{2.5};
      \draw[->, >=stealth, line width=2pt, epfldark,visible on=<2->] \xycoordcenterbelow{0}{1} -- \xycoordcenterabove{1}{-1} node [pos=.25, right] {$\mathbf{P}$}; % parallel
      \draw[->, >=stealth, line width=2pt, canard,visible on=<3->] \xycoordcenterbelow{0}{1} -- \xycoordcenterabove{0}{0} node [pos=.3, below] {$\mathbf{R}$}; % relative
      \draw[->, >=stealth, line width=2pt, leman,visible on=<4->] \xycoordcenterbelow{0}{1} -- \xycoordcenterabove{1}{0} node [pos=.2, above] {$\mathbf{L}$}; % leading-tone
    \end{tikzpicture}
    }
    \caption{The \emph{Tonnetz}.}
\end{figure}

    \end{column}
  \end{columns}
\end{frame}

\note[itemize]{
\small
\item While Neo-Riemannian transformations on the \emph{Tonnetz} are a useful analytical
  technique for a great deal of music, it has an obvious limitation:
\item It requires that the music under consideration is triadic.
\item There are a number of approaches trying to extend the classical paradigm, e.g.
  by also considering seventh chords.
\item But as an analyst, one can not know in advance which kinds of sonorities one will
  encounter with a new, maybe unknown piece.
\item Fortunately, there is a way out of this dilemma.
\item We can consider the \emph{Dual Tonnetz} in which not triads but the notes are in the focus.
\item This hexagonal reprentation allows us to visualise notes as they occur in a piece or in a section thereof,
  and to compare pieces with one another.
\item This approach is particularly relevant since we want to understand the historical development of tonality.
}

\begin{frame}[fragile]{\insertsectionhead}

  \begin{columns}
    \begin{column}{.5\linewidth}
      % The \emph{dual Tonnetz}

      \begin{itemize}
        \item<2-> \alert{diatonic} $\pm$ P5\\
        \vspace{1em}
          \includegraphics[width=\linewidth]{scores/diatonic.pdf}
        \item<3-> \alert{octatonic} $\pm$ m3, $\pm$ P5\\
        \vspace{1em}
          \includegraphics[width=.55\linewidth]{scores/octatonic.pdf}
        \item<4-> \alert{hexatonic} $\pm$ M3, $\pm$ P5\\
        \vspace{1em}
          \includegraphics[width=.45\linewidth]{scores/hexatonic.pdf}
      \end{itemize}
    \end{column}
    %
    \begin{column}{.5\linewidth}

    \begin{figure}[tbp]
    \centering
    \newcommand*{\hex}[1]{
      \def\r{0.55}
      \draw[black!6,fill=black!4] ($(#1)+(30:\r)$) -- ($(#1)+(90:\r)$) -- ($(#1)+(150:\r)$)  -- ($(#1)+(210:\r)$) -- ($(#1)+(270:\r)$) -- ($(#1)+(330:\r)$) -- cycle;
    }
    \begin{tikzpicture}[scale=1.7]
      % hexagon
      \hex{0,0}
      % notes
      \foreach \a/\l/\i/\p in {
        0/G/{$+$P5}/below,
        -60/{E$\flat$}/{$+$m3}/below,
        -120/{A$\flat$}/{$-$M3}/above,
        -180/F/{$-$P5}/above,
        -240/A/{$-$m3}/above,
        -300/E/{$+$M3}/below,
        1000/// % hack to avoid weird tikz label problem
      } {
        \ifthenelse{\a=1000}{}{
          \hex{\a:1}
          \draw (\a:0.15) edge[->,>=stealth,line width=1,shorten >=.5em] node[\p ,sloped,fill=white,fill opacity=0,text opacity=1] {\scalebox{0.8}{\i}} (\a:0.85);
          \node [circle,draw,black, fill=white, minimum size=2em] at (\a:1) {\l};
        }
      }
      % center
      \node [circle,draw,black, fill=white, minimum size=2em] at (0,0) {C};
    \end{tikzpicture}
    % \caption{Primary intervals with respect to C.}\label{fig:primary_intervals}
  \end{figure}
\end{column}
\end{columns}
\end{frame}

\begin{frame}{\insertsectionhead}

  \begin{columns}
    \begin{column}{.33\linewidth}
      \centering
      \alert{diatonic}

      \vspace{1em}

      \includegraphics[trim=0 100 0 100, clip, width=\linewidth]{img/bach.pdf}
      Bach, \emph{Prelude in C major}, BWV~846 (1722).
    \end{column}
    %
    \pause
    %
    \begin{column}{.33\linewidth}
      \centering
      \alert{hexatonic}

      \vspace{1em}

      \includegraphics[trim=0 100 0 100, clip, width=\linewidth]{img/liszt.pdf}
      Liszt, \emph{Lugubre gondola I}, S.~200/1~(1882).
    \end{column}
    %
    \pause
    %
    \begin{column}{.33\linewidth}
      \centering
      \alert{octatonic}

      \vspace{1em}

      \includegraphics[trim=0 130 0 100, clip, width=\linewidth]{img/scriabin.pdf}
      Scriabin, \emph{Prelude}, op.~74/2~(1914).
    \end{column}
  \end{columns}

  % \begin{figure}
  %   \centering
  %   \begin{subfigure}[t]{.3\textwidth}
  %     \includegraphics[width=\linewidth]{img/bach.pdf}
  %     \subcaption*{Bach, \emph{Prelude in C major}, WTC I, (1722).}
  %   \end{subfigure}
  %   \hfill
  %   \begin{subfigure}[t]{.3\textwidth}
  %
  %   \end{subfigure}
  %   \hfill
  %   \begin{subfigure}[t]{.3\textwidth}
  %
  %   \end{subfigure}
  %   \caption{Historical differences in note distributions.}
  % \end{figure}
\end{frame}

\note[itemize]{
\item Most importantly, this approach does not need a reduction to triads prior to analysis, since the notes can
  be taken from the score as they are written.
\item However, as in the case of the classical \emph{Tonnetz}, we consider octave related notes as equivalent.
}

\begin{frame}{\insertsectionhead}
  Computational model assumptions
  \begin{enumerate}[\color{epflred}1.]
    \item all notes are related to a \alert{tonal center}
    \item relations are given by (combinations of) \alert{intervals} on the \emph{Tonnetz}
  \end{enumerate}
\end{frame}

\begin{frame}{\insertsectionhead}
  \begin{enumerate}[\color{epflred}1.]
    \item \textbf{Initial computational model and analysis}\\\citet{Moss2019a}.~\citetitle{Moss2019a}.
    \item \textbf{Extensions of model}\\\citet{Lieck2020}.~\citetitle{Lieck2020}.
    \item \textbf{Extensions of analysis}\\\citet{Moss2020b}.~\citetitle{Moss2020b}.
  \end{enumerate}
\end{frame}


\begin{frame}{\insertsectionhead}

  \begin{figure}[h]
    \onslide<1->{
	\begin{subfigure}{0.44\linewidth}
		\centering
		\includegraphics[width=\textwidth]{img/tpc_dists.pdf}
	\end{subfigure}}
  %
  \hfill
  \onslide<2->{
	\begin{subfigure}{0.24\linewidth}
		\centering
		\includegraphics[width=\textwidth, trim=110 140 110 140,clip]{img/Schubert_90_2.pdf}
	\end{subfigure}}
  %
  \hfill
  \onslide<3->{
  \begin{subfigure}{0.29\linewidth}
		\centering
		\includegraphics[width=\textwidth]{img/tikz_Schubert_90_2.pdf}
	\end{subfigure}}
	\caption{F. Schubert, \emph{Impromptu}, op. 90/2 in E$\flat$ major.}
	% \label{}
\end{figure}

\onslide<4->{

Model finds six \alert{interval parameters} that best explain the notes in a piece.
\begin{itemize}
  \item only based on \alert{explicit} music-theoretical assumptions $\rightarrow$ criticism
  \item looking at all parameters for all pieces in the corpus shows \alert{historical trends}
\end{itemize}
}

\end{frame}

\begin{frame}
  % \begin{columns}
  %   \begin{column}{.4\textwidth}
  %     \begin{itemize}
  %       \item Computational Model to infer most likely interval usage~\citep{Lieck2020}
  %       \item Perfect fifths are highly relevant throughout history
  %       \item Major and minor third directions are explored abundantly in the 19th century
  %     \end{itemize}
  %   \end{column}
  %   %
  %   \begin{column}{.6\textwidth}
      \begin{figure}
        \centering
        \includegraphics[width=.65\textwidth]{img/bootstrapped_weights.pdf}
        \caption{Historical changes in interval usage on the \emph{Tonnetz}.}
        \label{}
      \end{figure}
  %   \end{column}
  % \end{columns}
\end{frame}

% \begin{frame}{\insertsectionhead}
%   \begin{enumerate}[\color{epflred}1.]
%     \item<1-> model finds \alert{best explanations} for relations between notes
%     \item<2-> only based on \alert{explicit assumptions} of the model $\rightarrow$ criticism
%     \item<3-> corpus studies allow to analyze \alert{changes of tonality} on a large scale
%   \end{enumerate}
% \end{frame}

% \begin{frame}{Summary and Prospects}
%   \begin{enumerate}[\color{epflred}1.]
%     \item \ldots
%     \item \ldots
%     \item Tonality tells only one part of the story. Many other aspects of music are relevant as well,
%     e.g. form, meter, instrumentation, tuning, social contexts, reception, \ldots
%   \end{enumerate}
% \end{frame}

% \section{Summary and Conclusion}

\begin{frame}{Summary}
  \begin{enumerate}[\color{epflred}1.]
  \item<+-> a small set of chords is shared by many composers
  \item<+-> this core makes up large portions of their harmonic language
  \item<+-> important aspects of chord distributions are not yet well-understood
  \item<+-> the line of fifths can be retrieved from data
  \item<+-> using the \emph{Tonnetz} reveals stylistic differences
  \item<+-> both can be expanded from the study of single pieces to entire repertoires
\end{enumerate}
\end{frame}

\begin{frame}{Conclusion}
  \textbf{Computational Music Analysis}
  \begin{enumerate}[\color{epflred}1.]\pause
    \item<+-> forces precise definitions and models
    \item<+-> allows to test music theoretical concepts on large corpora
    \item<+-> facilitates interactions with other areas within and beyond musicology
    \item<+-> requires reflection, critique, and interpretation
  \end{enumerate}
\end{frame}

% \againframe<8->{disciplines}

% \begin{frame}{}
%   \begin{figure}
%     \centering
%     \includegraphics[height=.9\textheight]{img/dcml.jpg}
%     \caption*{Digital and Cognitive Musicology Lab, Lausanne, 2019}
%   \end{figure}
% \end{frame}

\begin{frame}[standout]
  Thank you very much!
\end{frame}


%%% BACKMATTER %%%

\appendix

\begin{frame}[allowframebreaks]
  \renewcommand*{\bibfont}{\footnotesize}
  \printbibliography
\end{frame}

\end{document}
